% Options for packages loaded elsewhere
\PassOptionsToPackage{unicode}{hyperref}
\PassOptionsToPackage{hyphens}{url}
%
\documentclass[
]{article}
\usepackage{amsmath,amssymb}
\usepackage{iftex}
\ifPDFTeX
  \usepackage[T1]{fontenc}
  \usepackage[utf8]{inputenc}
  \usepackage{textcomp} % provide euro and other symbols
\else % if luatex or xetex
  \usepackage{unicode-math} % this also loads fontspec
  \defaultfontfeatures{Scale=MatchLowercase}
  \defaultfontfeatures[\rmfamily]{Ligatures=TeX,Scale=1}
\fi
\usepackage{lmodern}
\ifPDFTeX\else
  % xetex/luatex font selection
\fi
% Use upquote if available, for straight quotes in verbatim environments
\IfFileExists{upquote.sty}{\usepackage{upquote}}{}
\IfFileExists{microtype.sty}{% use microtype if available
  \usepackage[]{microtype}
  \UseMicrotypeSet[protrusion]{basicmath} % disable protrusion for tt fonts
}{}
\makeatletter
\@ifundefined{KOMAClassName}{% if non-KOMA class
  \IfFileExists{parskip.sty}{%
    \usepackage{parskip}
  }{% else
    \setlength{\parindent}{0pt}
    \setlength{\parskip}{6pt plus 2pt minus 1pt}}
}{% if KOMA class
  \KOMAoptions{parskip=half}}
\makeatother
\usepackage{xcolor}
\usepackage[margin=1in]{geometry}
\usepackage{color}
\usepackage{fancyvrb}
\newcommand{\VerbBar}{|}
\newcommand{\VERB}{\Verb[commandchars=\\\{\}]}
\DefineVerbatimEnvironment{Highlighting}{Verbatim}{commandchars=\\\{\}}
% Add ',fontsize=\small' for more characters per line
\usepackage{framed}
\definecolor{shadecolor}{RGB}{248,248,248}
\newenvironment{Shaded}{\begin{snugshade}}{\end{snugshade}}
\newcommand{\AlertTok}[1]{\textcolor[rgb]{0.94,0.16,0.16}{#1}}
\newcommand{\AnnotationTok}[1]{\textcolor[rgb]{0.56,0.35,0.01}{\textbf{\textit{#1}}}}
\newcommand{\AttributeTok}[1]{\textcolor[rgb]{0.13,0.29,0.53}{#1}}
\newcommand{\BaseNTok}[1]{\textcolor[rgb]{0.00,0.00,0.81}{#1}}
\newcommand{\BuiltInTok}[1]{#1}
\newcommand{\CharTok}[1]{\textcolor[rgb]{0.31,0.60,0.02}{#1}}
\newcommand{\CommentTok}[1]{\textcolor[rgb]{0.56,0.35,0.01}{\textit{#1}}}
\newcommand{\CommentVarTok}[1]{\textcolor[rgb]{0.56,0.35,0.01}{\textbf{\textit{#1}}}}
\newcommand{\ConstantTok}[1]{\textcolor[rgb]{0.56,0.35,0.01}{#1}}
\newcommand{\ControlFlowTok}[1]{\textcolor[rgb]{0.13,0.29,0.53}{\textbf{#1}}}
\newcommand{\DataTypeTok}[1]{\textcolor[rgb]{0.13,0.29,0.53}{#1}}
\newcommand{\DecValTok}[1]{\textcolor[rgb]{0.00,0.00,0.81}{#1}}
\newcommand{\DocumentationTok}[1]{\textcolor[rgb]{0.56,0.35,0.01}{\textbf{\textit{#1}}}}
\newcommand{\ErrorTok}[1]{\textcolor[rgb]{0.64,0.00,0.00}{\textbf{#1}}}
\newcommand{\ExtensionTok}[1]{#1}
\newcommand{\FloatTok}[1]{\textcolor[rgb]{0.00,0.00,0.81}{#1}}
\newcommand{\FunctionTok}[1]{\textcolor[rgb]{0.13,0.29,0.53}{\textbf{#1}}}
\newcommand{\ImportTok}[1]{#1}
\newcommand{\InformationTok}[1]{\textcolor[rgb]{0.56,0.35,0.01}{\textbf{\textit{#1}}}}
\newcommand{\KeywordTok}[1]{\textcolor[rgb]{0.13,0.29,0.53}{\textbf{#1}}}
\newcommand{\NormalTok}[1]{#1}
\newcommand{\OperatorTok}[1]{\textcolor[rgb]{0.81,0.36,0.00}{\textbf{#1}}}
\newcommand{\OtherTok}[1]{\textcolor[rgb]{0.56,0.35,0.01}{#1}}
\newcommand{\PreprocessorTok}[1]{\textcolor[rgb]{0.56,0.35,0.01}{\textit{#1}}}
\newcommand{\RegionMarkerTok}[1]{#1}
\newcommand{\SpecialCharTok}[1]{\textcolor[rgb]{0.81,0.36,0.00}{\textbf{#1}}}
\newcommand{\SpecialStringTok}[1]{\textcolor[rgb]{0.31,0.60,0.02}{#1}}
\newcommand{\StringTok}[1]{\textcolor[rgb]{0.31,0.60,0.02}{#1}}
\newcommand{\VariableTok}[1]{\textcolor[rgb]{0.00,0.00,0.00}{#1}}
\newcommand{\VerbatimStringTok}[1]{\textcolor[rgb]{0.31,0.60,0.02}{#1}}
\newcommand{\WarningTok}[1]{\textcolor[rgb]{0.56,0.35,0.01}{\textbf{\textit{#1}}}}
\usepackage{graphicx}
\makeatletter
\def\maxwidth{\ifdim\Gin@nat@width>\linewidth\linewidth\else\Gin@nat@width\fi}
\def\maxheight{\ifdim\Gin@nat@height>\textheight\textheight\else\Gin@nat@height\fi}
\makeatother
% Scale images if necessary, so that they will not overflow the page
% margins by default, and it is still possible to overwrite the defaults
% using explicit options in \includegraphics[width, height, ...]{}
\setkeys{Gin}{width=\maxwidth,height=\maxheight,keepaspectratio}
% Set default figure placement to htbp
\makeatletter
\def\fps@figure{htbp}
\makeatother
\setlength{\emergencystretch}{3em} % prevent overfull lines
\providecommand{\tightlist}{%
  \setlength{\itemsep}{0pt}\setlength{\parskip}{0pt}}
\setcounter{secnumdepth}{-\maxdimen} % remove section numbering
\ifLuaTeX
  \usepackage{selnolig}  % disable illegal ligatures
\fi
\IfFileExists{bookmark.sty}{\usepackage{bookmark}}{\usepackage{hyperref}}
\IfFileExists{xurl.sty}{\usepackage{xurl}}{} % add URL line breaks if available
\urlstyle{same}
\hypersetup{
  pdftitle={Week 3 Exercises},
  pdfauthor={Steven Simonsen},
  hidelinks,
  pdfcreator={LaTeX via pandoc}}

\title{Week 3 Exercises}
\author{Steven Simonsen}
\date{March 25, 2024}

\begin{document}
\maketitle

Please complete all exercises below. You may use any library that we
have covered in class UP TO THIS POINT.

\begin{enumerate}
\def\labelenumi{\arabic{enumi})}
\tightlist
\item
  Two Sum - Write a function named two\_sum()
\end{enumerate}

Given a vector of integers nums and an integer target, return indices of
the two numbers such that they add up to target.

You may assume that each input would have exactly one solution, and you
may not use the same element twice.

You can return the answer in any order.

Example 1:

Input: nums = {[}2,7,11,15{]}, target = 9 Output: {[}0,1{]} Explanation:
Because nums{[}0{]} + nums{[}1{]} == 9, we return {[}0, 1{]}. Example 2:

Input: nums = {[}3,2,4{]}, target = 6 Output: {[}1,2{]} Example 3:

Input: nums = {[}3,3{]}, target = 6 Output: {[}0,1{]}

Constraints:

2 \textless= nums.length \textless= 104 --109 \textless= nums{[}i{]}
\textless= 109 --109 \textless= target \textless= 109 Only one valid
answer exists.

\emph{Note: For the first problem I want you to use a brute force
approach (loop inside a loop)}

\emph{The brute force approach is simple. Loop through each element x
and find if there is another value that equals to target -- x}

\emph{Use the function seq\_along to iterate} In the function below, I
debated whether to use ``\textless{}'' or ``!='' in my if statment. I
ultimately went with ``\textless{}'' because the problem stated ``you
may not use the same element twice.''

\begin{Shaded}
\begin{Highlighting}[]
\FunctionTok{library}\NormalTok{(purrr)}

\NormalTok{nums\_vector }\OtherTok{\textless{}{-}} \FunctionTok{c}\NormalTok{(}\DecValTok{5}\NormalTok{,}\DecValTok{7}\NormalTok{,}\DecValTok{12}\NormalTok{,}\DecValTok{34}\NormalTok{,}\DecValTok{6}\NormalTok{,}\DecValTok{10}\NormalTok{,}\DecValTok{8}\NormalTok{,}\DecValTok{9}\NormalTok{)}
\NormalTok{target }\OtherTok{\textless{}{-}} \DecValTok{13}

\CommentTok{\#Assign initial loop\_work variable to 0 for comparison to hashloop later}
\NormalTok{loop\_work }\OtherTok{\textless{}{-}} \DecValTok{0}

\NormalTok{two\_sum }\OtherTok{\textless{}{-}} \ControlFlowTok{function}\NormalTok{(nums\_vector,target)\{}
  \CommentTok{\#Empty character vector assigned to variable loop vector for now}
\NormalTok{  loop\_vector }\OtherTok{\textless{}{-}} \FunctionTok{character}\NormalTok{()}
  \CommentTok{\#nested for statements using }
  \CommentTok{\#seq\_along to iterate over the num\_vector.}
    \ControlFlowTok{for}\NormalTok{(i }\ControlFlowTok{in} \FunctionTok{seq\_along}\NormalTok{(nums\_vector)) \{}
\NormalTok{      loop\_work }\OtherTok{\textless{}\textless{}{-}}\NormalTok{ loop\_work }\SpecialCharTok{+} \DecValTok{1}
      \ControlFlowTok{for}\NormalTok{(j }\ControlFlowTok{in} \FunctionTok{seq\_along}\NormalTok{(nums\_vector)) \{}
      \CommentTok{\#if the sum of element [i] and element [j] equal the target }
      \CommentTok{\#and aren\textquotesingle{}t the same number (could also use != to return}
      \CommentTok{\#the reverse of the result),}
      \CommentTok{\#concatenate and output the indices on a line}
\NormalTok{        loop\_work }\OtherTok{\textless{}\textless{}{-}}\NormalTok{ loop\_work }\SpecialCharTok{+} \DecValTok{1}
        \ControlFlowTok{if}\NormalTok{(i }\SpecialCharTok{\textless{}}\NormalTok{ j }\SpecialCharTok{\&\&}\NormalTok{ nums\_vector[i] }\SpecialCharTok{+}\NormalTok{ nums\_vector[j] }\SpecialCharTok{==}\NormalTok{ target) \{}
\NormalTok{          loop\_vector }\OtherTok{\textless{}{-}} \FunctionTok{c}\NormalTok{(loop\_vector, }\FunctionTok{paste}\NormalTok{(i, j))}
\NormalTok{      \}}
\NormalTok{    \}}
\NormalTok{    \}}
  \CommentTok{\#Return used outside of the for loop, but within function!}
  \FunctionTok{return}\NormalTok{(loop\_vector)}
\NormalTok{\}}
\CommentTok{\#run the function by assigning result to the function}
\NormalTok{result }\OtherTok{\textless{}{-}} \FunctionTok{two\_sum}\NormalTok{(nums\_vector, target)}
\CommentTok{\#print the result}
\FunctionTok{print}\NormalTok{(result)}
\end{Highlighting}
\end{Shaded}

\begin{verbatim}
## [1] "1 7" "2 5"
\end{verbatim}

\begin{Shaded}
\begin{Highlighting}[]
\CommentTok{\# Test code}
\CommentTok{\#expected answers}
\CommentTok{\#[1] 1 7}
\CommentTok{\#[1] 2 5}
\CommentTok{\#[1] 5 2}
\end{Highlighting}
\end{Shaded}

\begin{enumerate}
\def\labelenumi{\arabic{enumi})}
\setcounter{enumi}{1}
\tightlist
\item
  Now write the same function using hash tables. Loop the array once to
  make a hash map of the value to its index. Then loop again to find if
  the value of target-current value is in the map.
\end{enumerate}

\emph{The keys of your hash table should be each of the numbers in the
nums\_vector minus the target. }

\emph{A simple implementation uses two iterations. In the first
iteration, we add each element's value as a key and its index as a value
to the hash table. Then, in the second iteration, we check if each
element's complement (target -- nums\_vector{[}i{]}) exists in the hash
table. If it does exist, we return current element's index and its
complement's index. Beware that the complement must not be
nums\_vector{[}i{]} itself!}

In this function, I again used ``\textless{}'' instead of ``!='' in the
if statement for the same reason as above. Additionally, I returned the
indices after reading ``If it does exist, we return current element's
index and its complement's index.'' Therefore, my answers don't match
the expected answers - hopefully that's okay!

\begin{Shaded}
\begin{Highlighting}[]
\FunctionTok{library}\NormalTok{(hash)}
\end{Highlighting}
\end{Shaded}

\begin{verbatim}
## hash-2.2.6.3 provided by Decision Patterns
\end{verbatim}

\begin{Shaded}
\begin{Highlighting}[]
\NormalTok{nums\_vector2 }\OtherTok{\textless{}{-}} \FunctionTok{c}\NormalTok{(}\DecValTok{5}\NormalTok{,}\DecValTok{7}\NormalTok{,}\DecValTok{12}\NormalTok{,}\DecValTok{34}\NormalTok{,}\DecValTok{6}\NormalTok{,}\DecValTok{10}\NormalTok{,}\DecValTok{8}\NormalTok{,}\DecValTok{9}\NormalTok{)}
\NormalTok{target2 }\OtherTok{\textless{}{-}} \DecValTok{15}

\CommentTok{\#Assign hash\_work variable to 0}
\NormalTok{hash\_work }\OtherTok{\textless{}{-}} \DecValTok{0}


\NormalTok{two\_sum2 }\OtherTok{\textless{}{-}} \ControlFlowTok{function}\NormalTok{(nums\_vector2,target2)\{}
  \CommentTok{\#Assign h to an empty hash map}
\NormalTok{  h}\OtherTok{\textless{}{-}}\FunctionTok{hash}\NormalTok{()}
  \CommentTok{\#As noted above, assign each element\textquotesingle{}s value as a key and its index as a value to   the hash table.}
  \ControlFlowTok{for}\NormalTok{(i }\ControlFlowTok{in} \FunctionTok{seq\_along}\NormalTok{(nums\_vector2)) \{}
    \CommentTok{\#Add 1 to hash\_work for comparison to loop work at end. Use \textless{}\textless{}{-} global}
    \CommentTok{\#assignment operator to update the variable outside of the function}
\NormalTok{    hash\_work }\OtherTok{\textless{}\textless{}{-}}\NormalTok{ hash\_work }\SpecialCharTok{+} \DecValTok{1}
\NormalTok{    h[(nums\_vector2[i])] }\OtherTok{\textless{}{-}}\NormalTok{ i}

\NormalTok{  \}}
  \CommentTok{\#Empty character vector for iteration \#2}
\NormalTok{  result2 }\OtherTok{\textless{}{-}} \FunctionTok{character}\NormalTok{()}
  \ControlFlowTok{for}\NormalTok{(i }\ControlFlowTok{in} \FunctionTok{seq\_along}\NormalTok{(nums\_vector2)) \{}
    \CommentTok{\#Add again to hash\_work}
\NormalTok{    hash\_work }\OtherTok{\textless{}\textless{}{-}}\NormalTok{ hash\_work }\SpecialCharTok{+} \DecValTok{1}
    \CommentTok{\#As noted above, complement is equal to target less the nums\_vector[i]}
\NormalTok{    complement }\OtherTok{\textless{}{-}}\NormalTok{ target2 }\SpecialCharTok{{-}}\NormalTok{ nums\_vector2[i]}
    \CommentTok{\#The first part of the if statement checks to make sure the complement is }
    \CommentTok{\#less than nums\_vector2[i] to avoid duplication. }
    \CommentTok{\#The second part of the if statement checks to see if the complement is}
    \CommentTok{\#contained within the keys (names) of nums\_vector2[i]}
    \ControlFlowTok{if}\NormalTok{(complement }\SpecialCharTok{\textless{}}\NormalTok{ nums\_vector2[i] }\SpecialCharTok{\&\&}\NormalTok{ complement }\SpecialCharTok{\%in\%} \FunctionTok{names}\NormalTok{(h)) \{}
      \CommentTok{\#If found, append to result2 vector with the value (indices) of the}
      \CommentTok{\#complement, and the index of the original nums\_vector2 that corresponds }
      \CommentTok{\#with the value whos sum is equal to the target.}
      \CommentTok{\#as.character is needed because R converts keys to a character type.}
      \CommentTok{\#Since our values are numeric, I converted to character to ensure keys are}
      \CommentTok{\#treated as characters. Before adding this, I was getting very }
      \CommentTok{\#strange,unexpected errors.}
\NormalTok{      result2 }\OtherTok{\textless{}{-}} \FunctionTok{c}\NormalTok{(result2, }\FunctionTok{paste}\NormalTok{(h[[}\FunctionTok{as.character}\NormalTok{(complement)]], i))}
\NormalTok{    \}}
\NormalTok{  \}}
  \CommentTok{\#I made sure to use return OUTSIDE of the for loops, but still within function!}
  \FunctionTok{return}\NormalTok{(result2)}
\NormalTok{\}}

\CommentTok{\#Assign result2 to running the function}
\NormalTok{result2 }\OtherTok{\textless{}{-}} \FunctionTok{two\_sum2}\NormalTok{(nums\_vector2, target2)}
\CommentTok{\#Print the result2}
\FunctionTok{print}\NormalTok{(result2)}
\end{Highlighting}
\end{Shaded}

\begin{verbatim}
## [1] "1 6" "2 7" "5 8"
\end{verbatim}

\begin{Shaded}
\begin{Highlighting}[]
\CommentTok{\#comparison of hashwork vs loopwork. Much more efficient to use hash!}
\FunctionTok{print}\NormalTok{(hash\_work)}
\end{Highlighting}
\end{Shaded}

\begin{verbatim}
## [1] 16
\end{verbatim}

\begin{Shaded}
\begin{Highlighting}[]
\FunctionTok{print}\NormalTok{(loop\_work)}
\end{Highlighting}
\end{Shaded}

\begin{verbatim}
## [1] 72
\end{verbatim}

\begin{Shaded}
\begin{Highlighting}[]
\CommentTok{\#expected answers}
\CommentTok{\#[1] 10  5}
\CommentTok{\#[1] 8 7}
\CommentTok{\#[1] 9 6}
\CommentTok{\#[1]  5 10}
\CommentTok{\#[1] 7 8}
\CommentTok{\#[1] 6 9}
\end{Highlighting}
\end{Shaded}


\end{document}
